\chapter*{Introducción}\label{chapter:introduction}
\addcontentsline{toc}{chapter}{Introducción}

Desde hace varios años, la disminución de la productividad de la investigación y el desarrollo farmacéutico se ha identificado como un problema importante por su insostenibilidad en el tiempo. Esto es consecuencia tanto de lo que se tarda en introducir un nuevo fármaco, como los recursos económicos que se deben destinar para ello. Según el \textit{Tufts Center for the Study of Drug Development (CSDD)}, el costo de desarrollar un nuevo medicamento recetado con aprobación de comercialización es de aproximadamente $2600$ millones de dólares y solo uno de cada veinte medicamentos obtiene la aprobación para ensayos clínicos [\cite{alcim}].

La disminución de la productividad se traduce en un mayor riesgo de desarrollo y en un aumento de los precios de los medicamentos cuando llegan al mercado. Esto plantea una amenaza, no solo para la industria farmacéutica, sino también para la accesibilidad de medicamentos innovadores por parte de los pacientes.

\section*{Motivación}
Una de las estrategias utilizadas para la optimización del proceso de investigación y desarrollo farmacéutico es conocida como reposicionamiento de medicamentos.
El \textbf{reposicionamiento de medicamentos} (también llamado reutilización de medicamentos, reperfilado, redirección o redescubrimiento de drogas [\cite{repurposingconcept}]) consiste en la identificación de nuevos propósitos terapéuticos para medicamentos ya aprobados, más allá del alcance de su uso terapéutico original. Esta técnica ofrece varias ventajas sobre el desarrollo de fármacos completamente nuevos, donde se incluyen la posibilidad de acelerar el proceso de descubrimiento y reducir las tasas de fallas en las fases de desarrollo clínico y prueba [\cite{advantages}]. En particular, evita las evaluaciones de seguridad en modelos preclínicos y humanos, si estas se han completado ya para la indicación original y existe compatibilidad de dosis con la nueva indicación. Por consiguiente, lleva a costos generales de desarrollo potencialmente más bajos.

\section*{Antecedentes}

\section*{Problemática}
Dadas las desventajas que presenta el desarrollo de nuevos medicamentos por la vía regular, los enfoques para optimizar dicho proceso se encuentren bajo un escrutinio intenso. En el caso de la reutilización de medicamentos, tradicionalmente, las historias de éxito han resultado, principalmente, de hallazgos en gran medida oportunistas y fortuitos [\cite{drugfindings}]. Con vistas a hacer de este un proceso más sistemático, en los últimos años, se han desarrollado diversos enfoques computacionales. Sin embargo, ninguno de ellos ha llegado a ser totalmente efectivo. 

\section*{Objetivo}
\subsection*{Objetivo general}
El diseño e implementación de una propuesta de modelo de aprendizaje computacional que posibilite la reutilización de medicamentos. 

\subsection*{Objetivos específicos}

\begin{itemize}
    \item Realizar un análisis del estado del arte en materia de reposicionamiento computacional de medicamentos.
    \item Identificación de bases de datos con información sobre los medicamentos, así como sobre drogas ya reposicionadas, que puedan ser utilizadas en la propuesta.
    \item Diseñar e implementar un prototipo de modelo de aprendizaje de máquina a partir de las bases de datos y los vectores de análisis seleccionados, que sea capaz de predecir si una droga puede ser reposicionada con un determinado fin.
    \item Plantear una metodología de evaluación para el modelo que permita el análisis de los resultados una vez aplicado.
\end{itemize}

\section*{Estructura}