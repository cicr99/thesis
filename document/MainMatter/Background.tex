\chapter{Estado del Arte}\label{chapter:state-of-the-art}

\section{Reposicionamiento de medicamentos}

La reutilización de medicamentos, también conocida como reposicionamiento, es una estrategia que busca nuevos tratamientos médicos entre los medicamentos con licencia ya existentes, en lugar del desarrollo de nuevas moléculas [\cite{repurposingconcept}]. Esta técnica ofrece numerosas ventajas en contraste, en términos de desarrollo [\cite{redo}]:
\begin{itemize}
    \item Disponibilidad de datos de farmacocinética, farmacodinámica y posología
    \item Conocimiento de seguridad y toxicidad, incluidos los efectos adversos
    \item Experiencia clínica derivada de las indicaciones originales
    \item Bajo costo, en particular para el caso de medicamentos genéricos con múltiples fabricantes
    \item Comprensión de los mecanismos de acción y/o \textit{targets} moleculares.
\end{itemize}
Existen numerosos ejemplos de reposicionamiento de medicamentos de forma exitosa en el pasado, como es el caso de los siguientes [\cite{drexamples}, \cite{drexamples2}]:
\begin{description}
    \item[Aspirina] Comercializada inicialmente en 1899 como analgésico, es reposicionada por primera vez en la década de 1980 como un fármaco antiagregante plaquetario para prevenir eventos cardiovasculares
    \item[Talidomida] Prescrito en 1950 para el tratamiento de las náuseas y el insomnio en mujeres embarazadas y posteriormente aprobado para el tratamiento de la lepra.
    \item[Citrato de sildenalfilo] Desarrollado originalmente como un medicamento antihipertensivo, y luego reutilizado por Pfizer y comercializado como Viagra para el tratamiento de la disfunción eréctil.
    \item[Dimetilfumarato] Durante años, solo se conoció como un inhibidor de moho para proteger el cuero y como causa de alergias.Se ha comercializado como medicamento desde 1994, para tratar la psoriasis. Teniendo en cuenta su actividad antiinflamatoria, se propuso su uso en otra enfermedad autoinmune, la esclerosis múltiple. Se comercializó para esta última indicación a partir de 2013.
    \item[Finesterida] Originalmente recetado para tratar el agrandamiento de la próstata y luego, para curar la pérdida de cabello.
    \item[Clorpromazina] Desarrollada en un principio como antihistamínico, se comercializó posteriormente como sedante y antiemético.
    \item[Azidotimidina] Usada en el tratamiento del cáncer, ahora se reutiliza como una terapia contra el VIH
\end{description}

Como se puede apreciar, la reutilización de medicamentos no constituye una idea nueva en el campo de la medicina; sin embargo, como estrategia explícita de desarrollo, ha tenido auge en los últimos años. De hecho, datos obtenidos de \textit{PubMed} muestran que el número de publicaciones relacionadas con el reposicionamiento de medicamentos ha aumentado de forma exponencial desde 2004 [\cite{advances}]. Esto se debe, en parte, a la acumulación de grandes volúmenes de datos que permiten la implementación de enfoques computacionales. Las fuentes de información populares incluyen: registros de salud electrónicos, análisis de asociación de todo el genoma o perfiles de expresión génica, estructuras compuestas y otros datos de perfiles fenotípicos. La tabla (AÑADIR REFERENCIA A TABLA) muestra algunas de las bases de datos utilizadas con mayor frecuencia en la actualidad [\cite{databases}].

\section{Enfoques computacionales para el reposicionamiento de medicamentos}

El problema principal en el reposicionamiento de medicamentos es la detección de nuevas relaciones entre medicamentos y enfermedades. Para abordar dicho problema, se han desarrollado una variedad de enfoques que incluyen enfoques computacionales, enfoques experimentales biológicos y enfoques mixtos. [\cite{hanqing}]

Dado el crecimiento de la cantidad de datos biomédicos y farmacéuticos a gran escala disponibles públicamente, los enfoques computacionales que utilizan minería de datos, aprendizaje automático y análisis de redes se vuelven cada vez más importantes [\cite{jarada}]. Se pudiera asumir que los ensayos experimentales biológicos son más confiables y predictivos en comparación; sin embargo, las condiciones de experimentación, la cantidad de concentraciones probadas y la cantidad de repeticiones pueden influir en la precisión de los resultados. Además, en contraste, los enfoques computacionales son de menor costo y presentan menos limitaciones. [\cite{oprea}]

\subsection{Clasificación de los enfoques computacionales}
Los enfoques computacionales existentes pueden ser clasificados de diferentes formas de acuerdo a distintos puntos de vista. Por ejemplo, algunos investigadores los agrupan de acuerdo a las redes biológicas utilizadas: redes de interacción molecular, redes reguladoras de genes, redes metabólicas, redes de interacción entre proteínas, entre otras [\cite{lotfi}]. Otra clasificación, que se enfoca en los modelos y metodologías utilizadas durante el reposicionamiento, es: métodos basados en datos y basados en hipótesis [\cite{zou}], aunque esta clasificación general puede ser aplicada a cualquier enfoque computacional. De acuerdo a las metodologías centrales utilizadas, se han dividido también en tres categorías: enfoques basados en redes, enfoques de minería de texto y enfoques semánticos [\cite{koromina}]. Este trabajo abordará la última clasificación mencionada.

\subsubsection{Enfoques basados en redes}
Los métodos basados en redes se pueden dividir en: enfoques de agrupamiento (\textit{clustering}) y enfoques de propagación [\cite{hanqing}].

El propósito de los enfoques de agrupamiento basados en redes es el descubrimiento de nuevas relaciones de tipo fármaco-enfermedad o fármaco-objetivo en grupos más pequeños dentro de una red más grande. Están basados en la idea de que entidades biológicas (enfermedades, medicamentos, proteínas, etc.) que pertenecen al mismo módulo de redes, comparten características similares. Por tanto, utilizan algoritmos de \textit{cluster} para encontrar diferentes subredes(también conocidas como grupos o cliques).

Por otra parte, los enfoques de propagación están basados en el flujo de conocimiento adquirido previamente a través de las diferentes capas de una red. Estos parecen ser bastante efectivos para la identificación de interconexiones de interés [\cite{emig}]. De acuerdo a la forma en que se trata la red para la obtención de información, los métodos basados en propagación se separan igualmente en dos categorías: enfoques locales, centrados en secciones de la red, y enfoques globales, que examinan la red en su totalidad. Los enfoques de propagación local toman en cuenta una cantidad de información limitada de la red, por consiguiente, fallan en algunos casos a la hora de realizar predicciones correctas [\cite{emig}]. Por el contrario, los segundos presentan un mejor desempeño.

\subsubsection{Enfoques basados en minería de textos}
Los enfoques basados en minería de textos explotan la gran cantidad de literatura disponible, que es filtrada posteriormente para retener solo las fuentes relevantes y extraer de ellas conocimiento para un conjunto de términos biológicos de interés. El origen de estos métodos en el campo médico se describe como el método "ABC" [\cite{weeber}]. La idea consiste en que al examinar los conceptos A, B y C, donde el concepto A está relacionado con B y B con C, es posible que exista también una conexión entre A y C. Basados en este modelo, varios métodos de minería de texto han sido propuestos para encontrar posibles relaciones enfermedad-medicamento en la literatura. Igualmente, el desarrollo de técnicas de procesamiento de lenguaje natural (\textit{NLP - Natural Language Processing}), ha influido en el incremento de la implementación de herramientas para llevar a cabo este enfoque (véase [\cite{miningapps1}, \cite{miningapps2}])

Un estudio reciente por Andronis et al. [\cite{mining}] resume varios enfoques y fuentes en el campo de minería de textos para el reposicionamiento de medicamentos.

Las herramientas de minería de texto reducen la complejidad temporal del reposicionamiento de medicamentos y ayuda a los investigadores a verificar sus resultados experimentales devolviendo masivas cantidades de relaciones de entidades biológicas. Sin embargo, presenta ciertas desventajas, como es el problema de cobertura limitada. Este se debe a que entidades biomédicas o relaciones parcialmente importantes tales como mutaciones, objetivos y fármacos no son considerados.

\subsubsection{Enfoques semánticos}


\section{Grafos de conocimiento}
